\documentclass[11pt]{book}
\title{\textbf{Citogenetica e Ingegneria Cromosomica}}
\author{Simona Debilio}
\date{A.A. 2018/2019}
% manually added packages
\usepackage[margin=1in]{geometry}
\usepackage[italian]{babel}
% ====================================================
\usepackage{lmodern}
\usepackage{amssymb,amsmath}
\usepackage{ifxetex,ifluatex}
\usepackage{fixltx2e} % provides \textsubscript
\ifnum 0\ifxetex 1\fi\ifluatex 1\fi=0 % if pdftex
  \usepackage[T1]{fontenc}
  \usepackage[utf8]{inputenc}
\else % if luatex or xelatex
  \ifxetex
    \usepackage{mathspec}
    \usepackage{xltxtra,xunicode}
  \else
    \usepackage{fontspec}
  \fi
  \defaultfontfeatures{Mapping=tex-text,Scale=MatchLowercase}
  \newcommand{\euro}{€}
\fi
% use upquote if available, for straight quotes in verbatim environments
\IfFileExists{upquote.sty}{\usepackage{upquote}}{}
% use microtype if available
\IfFileExists{microtype.sty}{%
\usepackage{microtype}
\UseMicrotypeSet[protrusion]{basicmath} % disable protrusion for tt fonts
}{}
\ifxetex
  \usepackage[setpagesize=false, % page size defined by xetex
              unicode=false, % unicode breaks when used with xetex
              xetex]{hyperref}
\else
  \usepackage[unicode=true]{hyperref}
\fi
\hypersetup{breaklinks=true,
            bookmarks=true,
            pdfauthor={},
            pdftitle={},
            colorlinks=true,
            citecolor=blue,
            urlcolor=blue,
            linkcolor=magenta,
            pdfborder={0 0 0}}
\urlstyle{same}  % don't use monospace font for urls
\setlength{\parindent}{0pt}
\setlength{\parskip}{6pt plus 2pt minus 1pt}
\setlength{\emergencystretch}{3em}  % prevent overfull lines
\setcounter{secnumdepth}{0}

\date{}

\setlength{\parindent}{0pt}

\usepackage{graphicx}
\usepackage{caption}
\usepackage{subcaption}
\usepackage{siunitx}
\usepackage{wrapfig}

\begin{document}

\maketitle

\tableofcontents

\chapter{Programma}

Il programma delle lezioni prevede:

\begin{itemize}
\item 
Cenni storici;
\item 
Citogenetica classica e nuova;
\item 
Condensazione del DNA nel cromosoma eucariotico:

	\begin{enumerate}
	\item strutture ad anse e ruolo regolativo;
	\end{enumerate}

\item 
Metodi di localizzazione genica:

	\begin{enumerate}
	\item
	Ibridazione di cellule somatiche;
	\item 
	Ibridazione in situ;
	\end{enumerate}

\item 
Regolazione dell’espressione genica negli eucarioti multicellulari:

	\begin{enumerate}
	\item 
	Puffering (cromosomi politenici);
    \item 
    Trascrizione rDNA (cromosomi a spazzola);
    \item
    Determinazione del sesso;
    \item
    Compensazione del dosaggio nei mammiferi;
    \item
    Compensazione del dosaggio in Drosophila;
    \end{enumerate}

\item 
Imprinting genomico;
\item
Struttura del genoma degli eucarioti:

	\begin{enumerate}
    \item
    Progetti genoma;
    \item 
    Sequenze uniche e ripetute;
    \item
    Non coding RNA;
    \end{enumerate}
    
\item 
Citogenetica dei tumori:

	\begin{enumerate}
	\item
	Generalità;
    \item
    Esempi specifici: linfoma di Burkitt, LMC ph+, Retinoblastoma;
    \end{enumerate}
    
\item
Instabilità genomica:

	\begin{enumerate}
	\item
    Duplicazioni segmentali;
    \item
    Disordini genomici;
    \item
    Duplicazioni segmentali;
    \end{enumerate}
    
\item
Elementi funzionali del cromosoma eucariotico:

	\begin{enumerate}
    \item
    Telomeri;
    \item
    Origini di replicazione;
    \item
    Centromeri:
    \end{enumerate}
    	
		\begin{itemize}
        \item
        DNA centromerico;
        \item
        Proteine centromeriche;
        \item
        Paradosso del centromero;
        \item
        I centromeri nell’evoluzione;
        \end{itemize}
        
\item
Ricostruzione del cromosoma:
	
	\begin{enumerate}
    \item
    Cromosomi artificiali;
    \item
    Vettori cromosomici;
    \item
    Cromosomi artificiali di mammifero;
    \item
    Metodo bottom-up;
    \item
    Metodo bottom-down
    \end{enumerate}

\item
Terapia genica:

	\begin{enumerate}
    \item
    Terapia genica germinale;
    \item
    Terapia genica somatica;
    \item
    Terapia genica nei tumori;
    \item
    Problemi etici;
    \end{enumerate}

\item
Cellule staminali:

	\begin{enumerate}
    \item
    cellule staminali embrionali;
    \item
    cellule embrionali fetali e adulte;
    \item
    cellule iPS (inuced Pluripotent Stem Cells);
    \end{enumerate}
    
\item
Clonazione:

	\begin{enumerate}
    \item
    clonazione vegetale;
    \item
    conazione animale;
    \item
    clonazione terapeutica;
    \item
    clonazione riproduttiva;
    \item
    problemi etici.
    \end{enumerate}
\end{itemize}

\chapter{Introduzione}

La condensazione del DNA nel cromosoma eucariotico presenta un ruolo regolativo importante e rappresenta la base della regolazione epigenetica dell’espressione genica.

I cromosomi politenici di Drosophila sono, ad oggi, protagonisti della genetica molecolare. Drosophila continua infatti ad essere un ottimo modello per lo studio degli eucarioti poiché:
\begin{itemize}
\item
il numero di marcatori conosciuti è elevato;
\item
consente di seguire i diversi stadi di sviluppo;
\item
è un modello trasferibile negli eucarioti superiori.
\end{itemize} 

Un cromosoma politenico è un cromosoma gigante. I cromosomi politenici si formano in seguito a vari cicli di replicazione che producono molte copie (anche centinaia) di cromatidi fratelli che rimangono uniti.
La formazione dei cromosomi politenici ha la funzione di aumentare il volume cellulare ma può anche comportare un vantaggio metabolico dato che l'elevato numero di copie di geni permette un alto livello di espressione genica. In Drosophila melanogaster, per esempio, i cromosomi delle ghiandole salivari delle larve subiscono numerosi cicli di endoreplicazione, e questo consente di produrre grandi quantità di secreto prima dell'impupamento.
Al microscopio i cromosomi politenici mostrano caratteristici pattern di bande chiare e scure che possono essere utilizzati per identificare riarrangiamenti cromosomici e delezioni. Bande scure corrispondono spesso alla cromatina inattiva, mentre bande chiare si trovano di solito in aree con una maggiore attività trascrizionale.

La citogenetica è una branca della genetica. Il termine \emph{citogenetica} significa ``studio dei cromosomi'', si tratta quindi di una disciplina confinata alla descrizione dei cromosomi basata sull’utilizzo del microscopio.
In realtà oggi la citogenetica è descritta in maniera differente poiché, nel tempo, si è adeguata alle esigenze della biologia molecolare diventando strettamente complementare alle nuove tecnologie, note come post-genomica, che hanno raggiunto delle capacità descrittive, risolutive, e funzionali elevate; la citogenetica risulta quindi essere complementare alla genomica-post genomica. 

L’\emph{ingegneria cromosomica} rappresenta, invece, lo studio molecolare dei cromosomi utile per la costruzione di cromosomi artificiali.  

\section{Citogenetica classica e nuova citogenetica}
La citogenetica classica è una disciplina più vecchia della genetica, che nasce con la scoperta del microscopio e che su di esso si basa. Già i primi microscopisti, analizzando gli apici radicali delle piante e alcuni tessuti di organismi molto semplici, videro che vi erano delle cellule che subivano dei cicli di divisione cellulare e descrissero dettagliatamente i processi di meiosi e mitosi, prima ancora che nascesse la genetica.

La \textbf{citogenetica classica} consiste nello studio morfologico, al microscopio, del comportamento dei cromosomi alla mitosi e alla meiosi. 
Già i primi microscopisti si resero conto del fatto che vi fosse una grossa differenza tra ciò che avveniva nelle cellule mitotiche e in quelle meiotiche; essi infatti descrissero la segregazione dei cromosomi con una distribuzione equivalente del materiale genetico nelle cellule figlie durante la mitosi e scoprirono, invece, che vi erano alcuni tessuti in cui avveniva un dimezzamento del numero cromosomico a seguito della formazione dei gameti. Quanto scoperto a quei tempi risultò essere di difficile comprensione, ma risultò poi essere molto chiaro nel momento in cui la citogenetica e la genetica mendeliana si unirono in un’unica cosa che permise di capire che il comportamento dei cromosomi spiegava l’eredità mendeliana.

La citogenetica ha avuto una grossa spinta grazie all’evoluzione delle nuove tecnologie in campo umano, infatti già alla fine degli anni '50 questa disciplina risultò essere fondamentale per l’analisi del cariotipo umano normale e patologico, cioè per capire quali fossero le basi cromosomiche di alcune patologie e per la diagnosi delle stesse.

Fondamentali per poter identificare i cromosomi sono le tecniche di colorazione differenziale e le colture in vitro di cellule somatiche (i cromosomi risultano essere visibili durante la divisione cellulare). Se per i primi microscopisti lo studio dei tessuti meristematici degli apici radicali, trattandosi di tessuti in continua rigenerazione, non risultò essere difficile, lo studio dei cromosomi durante la divisione cellulare risultò invece più arduo a causa della difficoltà nel ritrovare tessuti in attiva rigenerazione spontanea nell’uomo. Tessuti di questo tipo ve ne sono, ma risultano essere difficilmente raggiungibili con tecniche non invasive (es. midollo osseo, tessuti tumorali e tessuti germinali).
Per questo motivo sono state sviluppate delle tecniche in grado di stimolare la proliferazione di tessuti differenziati che normalmente non si dividerebbero. Il tessuto di elezione è rappresentato dal sangue periferico, è infatti possibile indurre le cellule nucleate (linfociti) a proliferare; lo stesso può essere fatto effettuando dei piccoli espianti cutanei prelevando qualche mm\(^2\) di tessuto sottocutaneo.

La citogenetica classica è stata, fondamentalmente, la base delle leggi di Mendel. In passato vi fu un momento in cui la citogenetica classica, con l’avvento della biologia molecolare, sembrava fosse divenuta incapace di descrivere dettagli, soprattutto funzionali, che invece la biologia molecolare poteva arrivare a definire; veniva infatti considerata una ``cenerentola o un’ancella povera'' come quasi tutte le discipline morfologico-descrittive.
In realtà vi fu un’evoluzione sempre più veloce tale per cui la nuova citogenetica rappresenta sempre lo studio dei cromosomi ma con delle tecniche ad altissima risoluzione, fino ad arrivare a livelli di risoluzione comparabili a quelli della biologia molecolare delle analisi di sequenza.
La nuova citogenetica è quindi, a tutt’oggi, la base fisica della genetica, della biologia molecolare e della biologia strutturale perché permette di descrivere l’anatomia molecolare del cromosoma arrivando, addirittura, a ricostruirlo. 

Dal momento che la citogenetica possiede una visione d’insieme, risulterà essere molto importante perché rappresenterà il punto di partenza e di arrivo per l’analisi dei genomi complessi. Infatti, una volta stabiliti dei dettagli molecolari è importante poterli inserire in un contesto cromatinico definito, sarà proprio in questa fase che risulterà essere fondamentale la citogenetica.
Oggi è possibile fare una citogenetica di nuova generazione, basata sull’uso della citogenetica in assenza dei cromosomi e nota come citogenomica.

La patologia costituzionale e la citogenetica oncologica trovano un supporto insostituibile nello sviluppo di tecniche ad alta risoluzione che ad oggi permettono di fare delle descrizioni dettagliate delle diverse anomalie associate al cancro. Con la citogenetica è possibile stabilire, a livello molecolare, i punti di rottura coinvolti nei ri-arrangiamenti e, quindi, delimitare delle regioni critiche all’interno delle quali cercare oncogeni o oncosoppressori coinvolti nello sviluppo di quel particolare tipo di tumore; ciò risulterà avere un’importanza diagnostica e pronostica.

\section{Il cromosoma}
Il termina cromosoma è stato introdotto nel 1888 da Wilhelm Waldeier e classicamente risulta essere definito come \emph{un ``corpo colorato'' evidenziabile con coloranti basici durante la divisione cellulare} (mitosi, meiosi).

In realtà, ad oggi, la definizione appena data risulta essere superata poiché è definibile come cromosoma \emph{una qualsiasi molecola di acido nucleico che contenga l’informazione per la sintesi di tutte le proteine di un certo organismo, dotata di replicazione autonoma (origini di replicazione), circolare o lineare (nel secondo caso dotata di telomeri) e in grado di segregare fedelmente nelle cellule figlie (funzione espletata dal centromero).}

Il cromosoma eucariotico, durante la metafase, è costituito da due cromatidi fratelli (identici) legati al centromero. Ogni cromatidio è una molecola di DNA lineare e la ricombinazione tra cromatidi fratelli non determina un ri-arrangiamento cromosomico, ma solamente uno scambio di elementi.\\
In ogni eucariote diploide ogni cromosoma è presente in due copie, uno di origine materna e uno di origine paterna, tra loro omologhi. Gli omologhi non sono identici ma possono differire per varianti alleliche a tutti i loci.
Quanto appena detto risulterà essere particolarmente importante poiché rappresentante la base della replicazione sessuata e, quindi, la base della variabilità degli organismi.  

Il microscopio elettronico non aggiunge dettagli alla struttura del cromosoma poiché lo mostra semplicemente più grande e rappresentato da un caotico super-avvolgimento di una fibra non meglio identificata.\\
Il cromosoma eucariotico è formato da un braccio corto (\emph{p}, petit - la prima conferenza per la standardizzazione della nomenclatura citogenetica si svolse in Francia) e uno lungo (\emph{q} - ordine alfabetico).\\
Prendendo come riferimento il cariotipo umano, in ogni cromosoma la molecola è lunga \emph{da 45 a 285 milioni (10\(^6\)) pb}, contiene qualche migliaio di origini di replicazione, mediamente qualche migliaio di geni, un centromero e due telomeri.
Dopo la fase di sintesi del DNA ogni cromosoma è formato da due cromatidi fratelli che sono identici; ogni cromatidio è una doppia elica di DNA.

Il 1980 è la data di nascita delle tecnologie del DNA ricombinante che hanno dato l’avvio alla genetica e alla biologia molecolare. Nel 1981 Tomas Caskey, direttore del Baylor College of Medicine di Huston (USA), affermò su una famosa rivista scientifica che la citogenetica si sarebbe estinta entro 5 anni proprio perché di fronte alle enormi potenzialità delle nuove tecnologie sembrava che essa avesse esaurito le sue capacità descrittive, pur restando fondamentale per la parte diagnostica. 
Tuttavia, nel 1991, lo stesso Caskey, nella medesima rivista, affermò che la citogenetica avrebbe rappresentato l’onda del futuro. 

\subsection{Livelli di risoluzione}
\begin{itemize}
\item 
Un genoma umano aploide è costituito da 3 x 10\(^8\) pb (10\(^8\) = 100 milioni).
\item
Ogni cromosoma è costituito da 10\(^8\) bp.
\item
Una banda metafasica (unità visibile con colorazioni differenziali) presenta una dimensione di circa 10\(^7\) bp, mentre le dimensioni di una banda profasica saranno di 10\(^6\) pb; ci si avvicina sempre di più alle dimensioni del singolo gene.
Infatti lo sviluppo dei bandeggi ad alta risoluzione è stato un primo step utile a colmare quei vuoti presenti tra approcci citogenetici e molecolari.
\item
Le dimensioni di un gene risultano essere di 10\(^4\)-10\(^5\) bp.
\end{itemize}

L’esigenza di analizzare genomi sterminati, come quelli dei mammiferi, ha determinato lo sviluppo di tecniche molecolari che permettessero di fare analisi di segmenti genomici sempre più grandi.\\
Gli approcci classici permettevano di isolare e clonare segmenti di DNA nell’ordine del migliaio di pb, ma dovendo riempire di marcatori un genoma di 3 miliardi di pb, con soli pezzi di 1000 pb non si arriverà ad esprimere nulla. Sono quindi state sviluppate delle tecniche molecolari come il clonaggio in cromosomi artificiali batterici che permettono di isolare, maneggiare, sequenziare e descrivere segmenti che arrivano a 10\(^5\)-10\(^6\) bp, dimensioni delle bande che si analizzano con i bandeggi ad alta risoluzione.\\
La citogenetica molecolare di ultima generazione, in grado di estrarre il DNA e pettinarlo sui vetrini, permette di arrivare ad una risoluzione di alcune centinaia di pb. Le nuove tecnologie hanno quindi permesso di colmare quel vuoto tra approccio morfologico e molecolare. 
In realtà i due approcci sono complementari poiché quello molecolare è indiretto e necessita di un’interpretazione e di un’analisi statistica dei risultati, mentre il grosso vantaggio dell’approccio citogenetico è quello di poter confermare i risultati con un metodo visivo; sarà infatti possibile osservare al microscopio le sequenze di DNA e questo risulterà essere fondamentale perché è vero che la risoluzione è inferiore, ma si avrà una prova del fatto che le molecole che si era ipotizzato essere così organizzate lo saranno effettivamente e, trattandosi di un approccio fisico, sarà inoltre possibile orientare tali molecole, cosa che non sarà possibile fare interpretando la dimensione delle bande visibili su di un gel. 
Pertanto risoluzione e approcci differenti forniranno, insieme, un’informazione completa per capire la struttura di genomi complessi come quelli dei mammiferi.

\subsection{La teoria cromosomica dell’eredità}
La teoria cromosomica dell’eredità si basa sulla fusione delle informazioni ottenute dai primi microscopisti sul comportamento dei cromosomi alla meiosi e alla mitosi con le leggi di Mendel.\\
Nel 1887 Weismann, effettuando analisi microscopiche, descrisse il comportamento dei cromosomi affermando quanto segue:
\begin{itemize}
\item 
la sostanza nucleare alla mitosi si divide in maniera equivalente nelle cellule figlie;
\item
se si parla di cellule meiotiche, i gameti femminili perderanno metà della sostanza nucleare nel secondo globulo polare, che verrà poi rimpiazzata dallo spermatozoo;
\item
durante la formazione dei gameti il numero cromosomico viene dimezzato e la riproduzione sessuata si basa sulla fusione di due gameti aploidi;
\item
esiste un’identità genetica sostanziale tra uovo e spermatozoo. In realtà quanto appena affermato, come si vedrà più avanti, non rappresenta il vero ed è stato messo in discussione in seguito alla nascita della teoria dell’imprinting genomico;
\item
la riproduzione sessuata è un modo per incrementare e conservare la variabilità genetica.
\end{itemize}

Nel 1903-1904, dopo la riscoperta delle leggi di Mendel, grazie a due ricercatori, Sutton e Boveri, venne ridefinita la teoria cromosomica dell’eredità. Si effettuò infatti una rilettura del comportamento dei cromosomi descritto in precedenza sulla base delle leggi di Mendel: si fece una sintesi di quanto osservato da quest’ultimo e di quanto osservato in base al comportamento dei cromosomi arrivando a definire che vi era una perfetta sovrapponibilità tra comportamento dei geni (non ancora noti con questo nome poiché Mendel li definiva come particelle ereditarie) e cromosomi. Proprio da qui in poi nacque la genetica. Genetica e citogenetica progrediscono di pari passo poiché la seconda rappresenta la base fisica delle leggi di Mendel.

Bisogna sottolineare sottolineare l’importanza di alcuni esperimenti che dimostrano in maniera indiscutibile che, effettivamente, i geni si trovano sui cromosomi. Si tratta di esperimenti analoghi e contemporanei, svolti nel 1931. I primi sono stati fatti sul mais, mentre i secondi sulla Drosophila; si tratta di una dimostrazione eclatante della teoria cromosomica dell’eredità in quanto è la prova citologica che il crossing over è uno scambio fisico di materiale genetico tra cromosomi omologhi.

Il primo esperimento, svolto sul mais, sfrutta dei marcatori genetici, cioè mutazioni facilmente identificabili a livello morfologico sulla cariosside del mais, e marcatori citogenetici, che permettevano di distinguere i due cromosomi omologhi.\\
Si prende in considerazione un individuo eterogizote strutturale, avente i due cromosomi omologhi diversi per quanto riguarda la struttura: uno dei due omologhi presenta un marcatore eterocromatico, definito cnob, (nell’immagine rappresentato dalla palla nera), presente unicamente sui cromosomi del mais e, dalla parte opposta dello stesso, un allungamento; si tratta quindi di un cromosoma differente rispetto al suo omologo. 
I due cromosomi risultano anche essere differenti per quanto riguarda i geni presenti sugli stessi, quali: wax (ceroso, Wx) e colorless (indicante una cariosside colorata in giallo o sbiadita, C).

\begin{itemize}
\item  
\textbf{C} rappresenta la cariosside colorata:
\item
\textbf{c} rappresenta la cariosside senza colore;
\item
\textbf{Wx} rappresenta la cariosside cerosa;
\item
\textbf{wx} rappresenta la cariosside amilacea.
\end{itemize}

La cariosside lucida (Wx) sarà dominante su quella opaca (wx), mentre la cariosside colorata (C) risulterà essere dominante su quella sbiadita (c). 
Si ha quindi eterozigosi per quanto riguarda i geni Wx e C, poiché gli alleli dominanti e recessivi si trovano su cromosomi differenti, ed eterozigosi strutturale.

\clearpage
\begin{figure}
\centering
\begin{subfigure}{.5\textwidth}
  \centering
  \includegraphics[width=.4\linewidth]{./img/01_cromosomiMais.png}
  \caption{Prima del crossing-over}
  \label{fig:sub1}
\end{subfigure}%
\begin{subfigure}{.5\textwidth}
  \centering
  \includegraphics[width=.4\linewidth]{./img/02_crossingoverMais.png}
  \caption{Dopo il crossing-over}
  \label{fig:sub2}
\end{subfigure}
\caption{Cromosomi omologhi di mais presentanti marcatori genetici e citogenetici}
\label{fig:test}
\end{figure}

In seguito al crossing-over vi sarà la ricombinazione e il conseguente ottenimento di una progenie ricombinante. I ricombinanti, dal punto di vista fenotipico, avranno delle cariossidi:

\begin{itemize}
\item wx – C;
\item Wx –c.
\end{itemize}

Analizzando il cariotipo delle cariossidi ricombinanti sarà possibile osservare che i cromosomi hanno effettuato la ricombinazione poiché tra i due è avvenuto uno scambio fisico dei segmenti cromosomici; essi risulteranno infatti morfologicamente distinguibili.\\
L’esperimento appena spiegato rappresenta quindi una prova diretta che i geni si trovino sui cromosomi.

Un esperimento analogo venne effettuato nel 1931 da Stern, utilizzando il cromosoma X di Drosophila. I marcatori presi in considerazione furono:
\begin{itemize}
\item \textbf{carnetion (car)}, indicante il colore dell’occhio;
\item \textbf{bar (b)}, indicante la forma dell’occhio. 
\end{itemize}

In Drosophila l’allele dominante è indicato con il simbolo ``+'', mentre il recessivo con la sigla del gene coinvolto.
I cromosomi parentali presenteranno cromosomi X diversi: uno formato da due frammenti e l’altro con un prolungamento della regione centromerica. 
Anche in questo caso, come per il mais, l’individuo sarà un eterozigote strutturale ed eterozigote per carnetion e bar.

\begin{figure}[h]
\centering
\begin{subfigure}{.5\textwidth}
  \centering
  \includegraphics[width=.3\linewidth]{./img/03_cromosomiDrosophila.png}
  \caption{Prima del crossing-over}
  \label{fig:sub1}
\end{subfigure}%
\begin{subfigure}{.5\textwidth}
  \centering
  \includegraphics[width=.4\linewidth]{./img/04_crossingoverDrosophila.png}
  \caption{Dopo il crossing-over}
  \label{fig:sub2}
\end{subfigure}
\caption{Cromosomi omologhi di Drosophila presentanti marcatori genetici e citogenetici}
\label{fig:test}
\end{figure}

In seguito alla ricombinazione si otterrà un tipo ricombinante b/+ o car/+, ma anche cromosomi ricombinanti.

Questo esperimento dimostra, ancora una volta, che i cromosomi rappresentano la sede dei geni.


\subsection{Immagini}

\begin{figure}[h]
\centering
\includegraphics[scale=0.40]{img/04_cromo.png}
\caption{Immagine in contrasto di fase di cromosomi politenici non colorati}
\label{}
\end{figure}

\begin{figure}[h!]
\centering
\includegraphics[scale=0.40]{img/05_cromospazzola.png}
\caption{Cromosomi a spazzola o lamp-brush, si tratta di cromosomi giganti che hanno avuto una particolare importanza per lo studio della sintesi dell’RNA. Sono presenti in tutti gli organismi, ma sono ben distinguibili in Xenopus laevis}
\label{}
\end{figure}

\begin{figure}[h!]
\centering
\includegraphics[scale=0.50]{img/06_cromopachitenespazzola.png}
\caption{Cromosomi in pachitene, una delle fasi della prima profase meiotica. Anche in questo caso si tratta di cromosomi a spazzola, ma non altrettanto ben studiabili come quelli giganti di Xenopus}
\label{}
\end{figure}

\begin{figure}[h!]
\includegraphics[scale=0.50]{img/07_cromoumani.png}
\centering
\caption{Colorazione standard di cromosomi umani in metafase. I cromosomi sono colorati in modo uniforme con un colorante affine al DNA}
\label{}
\end{figure}

\begin{figure}[h!]
\centering
\includegraphics[scale=0.50]{img/08_cromobandeggi.png}
\caption{Particolarmente importante fu l’introduzione delle tecniche di bandeggio. Questa immagine mostra uno dei primi bandeggi storicamente sviluppati all’inizio degli anni `70 nel quale è possibile osservare delle regioni trasversali più o meno intensamente fluorescenti}
\label{}
\end{figure}

\begin{figure}[h!]
\centering
\includegraphics[scale=1.0]{img/09_cariotipo.png}
\caption{Successivamente vennero sviluppati altri bandeggi, questi furono fondamentali per l’identificazione dei cromosomi e per lo studio delle patologie cromosomiche. 
Per il riconoscimento dei cromosomi si utilizzano degli ideogrammi, questi rappresentano uno schema e vengono usati come riferimento per la descrizione di un cariotipo. Nell’ideogramma ogni banda rappresenta una rata(?) in modo tale che quando viene descritto un ri-arrangiamento cromosomico sarà possibile definire su che cromosoma ci si trova (braccio corto o lungo) e quali bande coinvolgano i punti di rottura.
Viene inoltre utilizzata una classificazione internazionale in modo che la descrizione del cariotipo sia universalmente comprensibile.
L’immagine a fianco mostra l’ideogramma di una metafase in cui la risoluzione è di sole 450 bande e una in cui invece la risoluzione è di 1700 bande. Non si tratta di una risoluzione elevata se si pensa che con i massimi livelli si arriva a distinguere fino a 3000 bande su un cariotipo umano. Sarà quindi possibile descrivere anche dei dettagli di riarrangiamenti molto piccoli, ciò risulterà essere fondamentale in molte patologie, ma soprattutto nel cancro}
\label{}
\end{figure}

\begin{figure}[h!]
\centering
\begin{subfigure}{.5\textwidth}
  \centering
  \includegraphics[width=.7\linewidth]{img/10_cromoallungati.png}
  \caption{}
  \label{fig:sub1}
\end{subfigure}%
\begin{subfigure}{.5\textwidth}
  \centering
  \includegraphics[width=1.1\linewidth]{img/11_cromoallungati.png}
  \caption{}
  \label{fig:sub2}
\end{subfigure}
\caption{Nell’immagine è possibile osservare l’utilizzo di una tecnica molecolare che permette di identificare determinate regioni cromosomiche con delle sonde molecolari. E’ possibile osservare una sonda marcata in verde e una in giallo. 
È possibile aumentare la risoluzione, potendo così cogliere più dettagli, utilizzando una tecnica che permette di stirare come degli elastici i cromosomi, nota come tecnica dei cromosomi meccanicamente allungati. }
\label{fig:test}
\end{figure}

\begin{figure}[h!]
\centering
\includegraphics[scale=0.80]{img/12_cromatinapettinata.png}
\caption{Vi sono poi delle tecniche che permettono di pettinare la cromatina sui vetrini, di stabilire come siano disposte le diverse sequenze, quali siano i rapporti relativi (cosa si trova a destra piuttosto che a sinistra) e anche la distanza tra le diverse sequenze}
\label{}
\end{figure}

\begin{figure}[h!]
\centering
\includegraphics[scale=0.70]{img/13_celltumorali.png}
\caption{Nell’immagine a fianco è possibile osservare un caso di identificazione, con approcci molecolari, di regioni amplificate nel cancro alla mammella. Ciò risulta essere fondamentale perché permette di capire quale sia il grado di progressione della patologia.
Il gene marcato in rosso risulta essere molto amplificato nelle cellule tumorali, rappresenterà quindi una fase avanzata della malattia. 
Questo tipo di indagine è fondamentale perché permette di capire il livello di gravità e progressione del tumore, ma anche di fare la stessa analisi al termine di un trattamento terapeutico per osservare se vi sia stata una regressione del tumore. }
\label{}
\end{figure}


\begin{figure}[h!]
\centering
\includegraphics[scale=0.60]{img/14_microdelezioni.png}
\caption{La citogenetica molecolare può essere utilizzata anche per identificare delle micro-delezioni che non sarebbero visibili neppure con i  bandeggi ad alta risoluzione. L’immagine mostrante questa tecnica fa riferimento alla sindrome di prader-willi, molto studiata a causa del fenomeno dell’imprinting genomico}
\label{}
\end{figure}

\begin{figure}[h!]
\centering
\includegraphics[scale=0.50]{img/15_cromoartificiale.png}
\caption{L’immagine a fianco mostra il lavoro che è stato fatto nel laboratorio della docente, in cui è stata fatta una mappa ad alta risoluzione della regione centromerica di un cromosoma umano, osservando quale fosse la sequenza e l’organizzazione dei diversi satelliti centromerici in una condizione normale e dopo aver costruito, attraverso delezioni successive, un cromosoma artificiale. 
Con i diversi gradi di risoluzione è possibile osservare i rapporti tra le diverse sequenze e la presenza/assenza di alcune delle stesse, marcate in fluorescenza con colori differenti}
\label{}
\end{figure}


\begin{figure}[h!]
\centering
\includegraphics[scale=1.00]{img/16.png}
\caption{Nell’immagine sono stati analizzati i cromosomi allungati meccanicamente. In questo caso ciò che è interessante da osservare è non solo la struttura del centromero, ma anche la presenza di sequenze telomeriche. In questo caso era stato clonato, all’interno di un sito di clonaggio di un cromosoma artificiale, un gene marcatore (in verde nella figura) che risulta essere presente in 5 copie integrate}
\label{}
\end{figure}

\begin{figure}[h!]
\centering
\includegraphics[scale=1.00]{img/17.png}
\caption{L’immagine a fianco mostra l’analisi di mutazioni del gene codificante per la distrofina su fibra di cromatina estesa.
Si tratta di un’immagine di un lavoro derivante dalla letteratura che prende in considerazione la distrofia di Duchenne. Il gene della distrofina è il gene più grande fino ad oggi clonato del genoma umano che quando mutato determina, appunto, la sindrome di Duchenne, una malattia invalidante abbastanza frequente. Risulterà quindi importante caratterizzare le diverse mutazioni sia per la diagnosi che per tentare degli approcci terapeutici mediante terapia genica.
In questo caso è stata costruita la mappa delle delezioni del gene della distrofina in diversi pazienti attraverso tecniche di citogenetica molecolare ad alta risoluzione grazie alle quali è possibile identificare i segmenti mancanti nelle differenti situazioni; ciò risulterà essere importante perché permetterà poi di vedere quali siano le delezioni più frequenti e, quindi, effettuare una diagnostica mirata}
\label{}
\end{figure}

\begin{figure}[h!]
\centering
\includegraphics[scale=0.80]{img/18.png}
\caption{L’immagine a fianco mostra l’utilizzo di tecniche molecolari a più colori che permettono di colorare i cromosomi in modi differenti e di effettuare delle analisi in cui i ri-arrangiamenti cromosomici, anche molto complessi, potranno essere identificati grazie proprio allo scambio di colori tra i cromosomi. 
Se ogni cromosoma possiede un colore proprio, sarà possibile trovare un cromosoma arlecchino formato da diversi tasselli di provenienza nota. 
Quando si ha a che fare con ri-arrangiamenti che possono avere anche 10-11 punti di rottura, è chiaro che un bandeggio non permette di capire quali siano i punti di rottura dei cromosomi coinvolti; queste nuove tecniche, note come sky fish o spectral karyotyping, permetteranno invece di vedere anche ri-arrangiamenti molto complessi e di indirizzare l’analisi e la ricerca degli esatti punti di rottura}
\label{}
\end{figure}

\begin{figure}[h!]
\centering
\includegraphics[scale=0.70]{img/19.png}
\caption{A fianco sono mostrate delle immagini di immunofluorescenza del fuso mitotico e dei cromosomi che permettono di osservare se vi siano delle anomalie di segregazione dei cromosomi rilevabili attraverso modificazioni della struttura del fuso mitotico. 
Generalmente si tratta di anomalie riguardanti il centromero}
\label{}
\end{figure}

\begin{figure}[h!]
\centering
\includegraphics[scale=0.60]{img/20.png}
\caption{Ad oggi vi sono delle tecniche, risultanti fondamentali, che permettono di studiare i cromosomi nel nucleo interfasico. Nell’immagine è infatti possibile osservare dei nuclei
interfasici dove ciascuna banda di una particolare coppia di cromosomi è colorata in modo differente, sarà così possibile osservare l’organizzazione nel nucleo interfasico dei diversi distretti cromosomici. Ciò risulta essere molto importante perché è stato scoperto che la distribuzione tridimensionale dei vari domini cromosomici nel nucleo interfasico è direttamente correlata alla funzione degli stessi e che modificazioni della distribuzione si hanno nel corso del ciclo cellulare e possono anche essere indice di particolari perturbazioni dello stesso o di determinati pathway metabolici}
\label{}
\end{figure}

\begin{figure}[ht!]
\centering
\includegraphics[scale=0.70]{img/21.png}
\caption{Queste tecniche permettono di fare un bandeggio cromosomico a più colori e vedere, nel nucleo interfasico, come sono organizzati i diversi distretti cromosomici. 
Questi approcci prevedono inoltre delle analisi tridimensionali del nucleo interfasico. È possibile arrivare a misurazioni e a simulazioni tridimensionali della disposizione di tutti i domini, queste risulteranno essere particolarmente importanti per fare delle deduzioni sull’importanza della posizione di particolari regioni cromosomiche nel nucleo interfasico e del loro significato funzionale}
\label{}
\end{figure}

\clearpage
La citogenetica del futuro è, fondamentalmente, la citogenomica; quest’ultima non vuole sostituire la citogenetica morfologica al microscopio, si tratta infatti di una disciplina che prevede l’utilizzo dei micro-array. Si parla quindi di citogenomica poiché rappresenterà un tipo di citogenetica fatta mediante approcci di genomica. 

\chapter{L'organizzazione del DNA}

A differenze dei procarioti, negli organismi eucarioti il grosso della regolazione dell'espressione genica avviene grazie a modificazioni del grado di condensazione del DNA, per questa ragione è molto importante conoscerne l'organizzazione.

In questo corso verranno approfonditi quei meccanismi che, modificando il grado di condensazione della cromatina (non è corretto parlare solo di DNA in quanto questo è complessato con le proteine istoniche), modulano la funzione delle cellule sia durante il ciclo cellulare che durante lo sviluppo e il differenziamento.
Il primissimo livello di regolazione dell’espressione genica negli eucarioti è dunque di carattere epigenetico. È evidente che siano importanti le sequenze regolatrici (promotori, enhancers, silencers, ecc), tuttavia l’attività di queste sequenze è modulata dalla loro accessibilità e quindi dal grado di condensazione della cromatina.

Prendiamo come punto di riferimento il genoma umano: nell’uomo la quantità totale in peso/massa di DNA è enorme, \emph{300 milioni bp} in aploidia. Se facciamo il conto, moltiplicando i 300 milioni per l’ingombro di ogni singola base, arriviamo ad una quantità di DNA pari a \textbf{1,80 metri} di lunghezza. Abbiamo dunque un'enorme sproporzione tra la dimensione del DNA e la dimensione del comparto in cui esso deve essere contenuto, e cioè il nucleo, il cui ordine di gradezza è quello dei micrometri. 

Si pone dunque il problema di come riuscire a condensarlo affinchè possa stare al suo interno: il DNA è una molecola molto sottile che potrebbe essere contenuta nel nucleo semplicemente comprimendola al suo interno ma, dal momento che circa ogni 24h il DNA di una cellula deve essere ripartito correttamente in due cellule figlie, è evidente che questo non può essere una matassa caotica ma deve essere rapidamente districabile. 
Il DNA raggiunge il suo massimo grado di condensazione durante la divisione cellulare (condensazione massina in metafase) e i cromosomi diventano dei ``gomitolini'' ben distinti con un livello di condensazione tale da renderli morfologicamente ben delineati e facilmente identificabili e classificabili al microscopio.

Il DNA è diviso in cromosomi, e negli essere umani la lunghezza totale di 1,80 metri è divisa in 23 coppie di cromosomi. Nell'organizzazione del DNA è poi molto importante che regioni che sono in qualche modo funzionalmente correlate ma linearmente molto distanti tra di loro, debbano potersi trovare vicine in certi momenti del ciclo cellulare, dello sviluppo e/o del differenziamento, in modo da poter essere espresse o replicate (replicazione e espressione sono fenomeni concertati) in maniera coordinata. È dunque fondamentale l'esistenza di meccanismi che permettano a tratti di DNA che, se consideriamo la molecola lineare, sarebbero molto distanti tra di loro di trovarsi in condizione di poter rispondere agli stessi segnali regolativi: si tratta fondamentalmente di meccanismi epigenetici di regolazione dell’espressione genica.

Qual è la soluzione ai problemi che abbiamo esposto? Com'è possibile che tutto il DNA entri nel nucleo?
Per risolvere questo problema il DNA viene condensato in modo ordinato, secondo una sequenza gerarchica di superavvolgimenti.
In generale il modello che viene seguito è il \textbf{modello a spirale}, un modello molto frequente in biologia e in tutto il mondo naturale in quanto, dal punto di vista fisico, è un modello energeticamente economico, ovvero è il modello che permette di ottenere forme ordinate con il minor dispendio di energia.


Vediamo alcuni rapporti tra dimensione del genoma e dimensione del comparto in cui questo è contenuto:
\begin{itemize}
\item Il \textbf{virus del mosaico del tabacco} è un virus filamentoso la cui dimensione è di 0.008 x 0.3  micron. Ha un genoma la cui lunghezza è di 6.4 Kb che corrispondono a 2 m. In questo caso non c’è una sproporzione così esagerata tra dimensioni del comparto/cellula/virus e dimensioni del genoma.
\item Il \textbf{Fago T4}, una cellula icosaedrica, ha una dimensione di 0.065 x 0.10 microm. Qui la sproporzione è già più evidente perché il genoma è di 170 Kb, per una lunghezza di 55 m.
\item \textbf{E. coli} è un batterio cilindrico tra i più conosciuti in quanto ampiamente utilizzato come organismo modello, della dimensione di 1.7 x 0.65 m. Il suo genoma è costituito da 4.2 x 103 Kb e ha una lunghezza di 1.3 mm (non più m).
\item Il \textbf{genoma umano aploide} è di \textbf{3x10\(^9\) Kb}, un genoma diploide è di 6 x 10\(^9\) Kb pari ad 1.80 metri. Il DNA umano è poi suddiviso in 46 cromosomi, ognuno dei quali è costituito da una molecola di DNA di circa 8 cm (la dimensione dei cromosomi è variabile), mentre il nucleo somatico sferico ha una dimensione di 6 micron (i nuclei delle cellule uovo sono molto più grandi).
\end{itemize}

La sproporzione tra la dimensione del genoma e quella del comparto in cui deve essere contenuto è evidente in tutti gli organismi e in particolare negli eucarioti superiori.
Una cosa fondamentale è che la cromatina, nelle cellule eucariotiche, viene trascritta e replicata in interfase.

Dal punto di vista energetico invece la fase più dispendiosa per la cellula è quella della divisione vera e propria, quando tutto il macchinario cellulare è concentrato sulla ipercondensazione della cromatina, sulla costruzione del fuso mitotico, sulla costruzione del cinetocore e su tutti quei processi che permettono la corretta separazione dei cromosomi nelle cellule figlie. Quando la cellula si divide la cromatina è ipercondensata: non viene né trascritta né replicata, salvo rarissime eccezioni.
Tutta l’attività metabolica della cellula invece si svolge durante l’interfase, quando il DNA è sì condensato perché deve stare nel nucleo, ma il grado di condensazione della cromatina è tale da consentire la funzionalità e la regolazione dell’espressione genica. 


\section{La cromatina}
Quando ci si riferisce agli eucarioti non bisognerebbe mai parlare di DNA ma di cromatina, in quanto normalmente il DNA è complessato con le proteine istoniche e solo transitoriamente è ``nudo''.

Gli istoni sono le proteine più abbondanti nel nucleo di una cellula eucariotica (rapporto in massa tra DNA e istoni di 1 a 1).

La cromatina viene divisa in \emph{eucromatina} ed \emph{eterocromatina}. Questa divisione è basata su osservazioni di microscopia ottica e si riferisce alla colorabilità con coloranti basici di diverse regioni del nucleo interfasico, distinguiamo:
\begin{itemize}
\item \emph{regioni più intensamente colorabili} che sono quelle più condensate durante l’interfase e sono le \textbf{regioni eterocromatiche}; 
\item \emph{regioni meno colorabili}, che sono quelle meno condensate durante l'interfase e sono le \textbf{regioni eucromatiche}.
\end{itemize}

Queste sono in realtà delle grosse generalizzazioni perché si basano soltanto su analisi morfologiche al microscopio ottico. In realtà, le regioni eucromatiche, come anche quelle eterocromatiche, contengono al loro interno rispettivamente regioni eterocromatiche e regioni eucromatiche: si tratta quindi di sub-regioni.

A sua volta l’eterocromatina si divide in:
\begin{itemize}
\item \textbf{eterocromatina costitutiva}. Questa è un'eterocromatina di tipo costituzionale, sempre presente a prescindere dalle fasi dello sviluppo, del differenziamento e del ciclo cellulare in cui una cellula di una certa specie si trova. Rappresenta una \emph{caratteristica intrinseca di una certa regione del genoma}. Questo tipo di eterocromatina è sempre presente in una determinata regione del genoma in tutte le cellule di tutti gli organismi della stessa specie.\\
In realtà ad oggi questo concetto è stato ampiamente messo in discussione.
Esempio eclatante: quando parliamo di eterocromatina costitutiva tipicamente ci vengono in mente le regioni centromeriche di tutti i cromosomi, ma oggi sappiamo che la regione centromerica non è completamente eterocromatica anzi, ad essere eterocromatica è la regione \emph{pericentromerica} che forma una sorta di barriera e che contiene al suo interno il core funzionale centromerico trascrizionalmente competente.
\item \textbf{eterocromatina facoltativa}. Questa eterocromatina è presente soltanto in certe condizioni, può variare, e quindi non caratterizza la struttura di una certa regione cromosomica. 
Questo tipo di eterocromatina può riguardare certe regioni del genoma solo in alcune cellule di alcuni organismi della stessa specie oppure può riguardare soltanto uno degli omologhi.\\
L’esempio meglio studiato di eterocromatina facoltativa è il cromosoma X inattivo nelle cellule somatiche delle femmine di mammifero (\textbf{N.B.:} l'inattivazione dell'X non riguarda solo l’uomo ma tutti i mammiferi e, più in generale, tutti gli organismi con eteromorfismo dei cromosomi sessuali).
Questo è l’esempio meglio studiato di eterocromatina facoltativa, ma in natura ce ne sono molti altri (per esempio in alcuni insetti vengono inattivati tutti i cromosomi di origine paterna nelle cellule somatiche dei maschi, e questo è un meccanismo di determinazione del sesso).
\end{itemize}

\begin{figure}[h!]
\centering
\includegraphics[scale=0.70]{img/22_cromatina.png}
\caption{Diversi livelli di condensazione del DNA. Possiamo vedere sia una rappresentazione schematica che l’immagine al microscopio elettronico. Inoltre nell’immagine si vede anche la dimensione, lo spessore, della fibra a cui ci riferiamo: il confronto di spessore delle fibre con ordine sempre superiore di superavvolgimento ci dà un’idea del grado di condensazione.}
\label{livelli_condensazione}
\end{figure}

Il ciclo di condensazione del DNA è rappresentato nell’immagine \ref{livelli_condensazione}, dove abbiamo una divisione molto netta tra ciò che riguarda l’interfase e ciò che riguarda invece le fasi successive all'interfase che portano alla divisione cellulare. La parte dell’interfase è quella che ci interessa di più perché è la fase in cui la cromatina funzionale è sì superavvolta ma in modo tale da poter comunque permettere la regolazione dell’espressione genica.\\
A partire dalla \emph{doppia elica} di DNA che ha un diametro di \textbf{20 \si{\angstrom}} (pari a 2 nm), passiamo alla struttura di ordine superiore che è la \emph{collana di perle} che si vede molto bene al microscopio elettronico e che ha un superavvolgimento di 7 volte per una dimensione di \textbf{11 nm}. Un’ulteriore condensazione di 6 volte porta dalla collana di perle alla \emph{struttura a solenoide} che ha un diametro di \textbf{30 nm}; si arriva poi alla \emph{struttura ad anse} che prevede un superavvolgimento di altre 40 volte per una dimensione di \textbf{300 nm} e una superavvolgimento totale di 1.600 volte per quanto riguarda la condensazione di DNA in interfase.\\
Le fasi successive seguono di nuovo un modello ad elica: abbiamo un superavvolgiemento di 5x5 volte che porta al cromosoma profasico e poi al cromosoma metafasico che raggiunge il massimo della condensazione.

Il modello seguito per il superavvolgimento è sempre quello a spirale:
\begin{itemize}
\item La doppia elica è un modello a spirale;
\item La collana di perle è un modello a spirale perché il DNA su ogni nucleosoma si superavvolge con due spire;
\item Il modello a solenoide è un modello a spirale perché in ogni giro del solenoide ci sono 6 perle della collana di perle;
\item Il \textbf{modello a loop} invece \underline{non è} un modello a spirale ma è un \emph{modello variabile}: a variare è sia la dimensione dei loop che il loro grado di affastellamento, di condensazione. È questa la parte più importante su cui ci soffermeremo perché ha un significato regolativo fondamentale.
\end{itemize}

Il processo di condensazione del DNA porta dunque da una fibra di 1,8 m a cromosomi singoli in cui mediamente il DNA è lungo 8 cm per cromosoma (in realtà, al microscopio, i cromosomi metafasici umani, ipercondensati, hanno una dimensione che varia da 2 a 9 micron circa).

Parliamo ora della struttura della cromatina: con questo termine ci si riferisce infatti alla fibra di DNA complessata con le proteine istoniche (basiche) con un rapporto in massa di 1:1.
Gli istoni sono in assoluto le proteine più abbondanti nel nucleo di una cellula eucariote.
La fibra elementare, e cioè la collana di perle, ha un diametro di circa 10-11 nm e deve il suo nome alla ripetizione assolutamente regolare di strutture globulari chiamate \textbf{nucleosomi}. Ogni nucleosoma/perla è poi legato all'altro da un segmento di DNA chiamato \textbf{DNA linker}.

I nucleosomi sono costituiti da un core proteico, che forma la struttura glomerulare, formato da un ottamero di istoni. Gli istoni sono 5: \textbf{H1}, \textbf{H2a}, \textbf{H2b}, \textbf{H3} e \textbf{H4}.
A formare il nucleosoma partecipano solo gli istoni\emph{ H2A, H2B, H3 e H4} con due subunità per tipo.\\
In questo momento l’istone H1 non partecipa ancora alla condensazione.

Su ogni nucleosoma si superavvolgono \textbf{2 spire di DNA} per un totale di circa \textbf{140-150 pb}.\\
Le perle poi sono unite tra loro da un ``filo'', che è il \textbf{DNA linker}, lungo circa \textbf{50 pb}.

\begin{wrapfigure}{r}{0.3\textwidth}
    \includegraphics[width=0.27\textwidth]{img/23_histoneH1.png}
  \caption{Istone H1}
\end{wrapfigure}

L’\textbf{istone H1} invece si trova all’esterno rispetto all’ottamero e il suo compito è quello di collegare tra loro nucleosomi adiacenti determinando il superavvolgimento di ordine superiore, quello che porta alla formazione del \emph{solenoide}: in ciascun giro del solenoide si trovano \textbf{6 nucleosomi}, per portare alla struttura della fibra di 30 nm.

L’istone H1 nell’immagine di fianco è rappresentato come una specie di girino, con una testa (estremità C-terminale), una coda (estremità N-terminale) e una pancia. 
La pancia fornma la regione che interagisce con il DNA che si trova superavvolto sul nucleosoma, mentre la testa e la coda servono per i legami proteina-proteina che permettono la contrazione della fibra, in quanto l’istone H1 è legato all’esterno del nucleosoma attraverso un'interazione DNA-proteina.

Fino a questo livello di superavvolgimento c’è un motivo monotono ricorrente del modello a spirale che è più o meno uguale in tutti gli organismi e che segue questo ordine gerarchico. È un modello difficilmente modulabile fermo restando che, quando il DNA si deve replicare e deve essere trascritto, le proteine istoniche si devono staccare molto rapidamente per liberarlo.
Non è un caso che non abbiamo mai parlato di legami covalenti: queste sono tutte interazioni di tipo elettrostatico, fondamentalmente ponti H, proprio perché richiedono una bassa energia, sono facilmente reversibili e facilmente ricostruibili nel momento in cui si è compiuto il processo di replicazione e trascrizione.

\textbf{[DOMANDA:} il modello a zig-zag per quanto riguarda la fibra a 30nm non viene considerato? Il modello a zig-zag si ha quando il DNA linker ha una maggiore dimensione. Sono tutte variazioni sul tema: ci sono anche altri modelli più o meno validi e più o meno dimostrati, ma il modello più attendibile è quello a solenoide. Il modello a zig-zag si riferisce forse ad una sorta di ibrido tra il modello a solenoide e la struttura a loops.\textbf{]}

\subsection{La struttura a loops}

\begin{wrapfigure}{r}{0.3\textwidth}
    \includegraphics[width=0.30\textwidth]{img/25_backbone.png}
  \caption{Istone H1}
\end{wrapfigure}

È possibile trattare un cromosoma metafasico con una sostanza che permette di staccare gli istoni e che quindi permette la completa decondensazione (se tolgo gli istoni tutte le strutture di  ordine inferiore ovviamente si perdono). Osservando al microscopio elettronico il cromosoma metafasico trattato con questa sostanza vediamo che resta uno scheletro, chiamato \emph{``back bone''}, che rispecchia la morfologia del cromosoma metafasico. Tuttavia, a partire da questa sorta di colonna vertebrale che rimane, si vedono delle estensioni/estroflessioni che formano una sorta di alone intorno ad esso. Aumentando l’ingrandimento si può vedere come questo alone che si estroflette dal back bone sia costituito da fibre molto sottili che rappresentano delle anse continue: le anse si estroflettono a partire dallo scheletro per poi riattaccarvisi.\\
Queste analisi al microscopio elettronico sono la dimostrazione morfologica della struttura ad anse della cromatina, una struttura molto variabile e non sempre presente. Questa struttura, nonostante vari tra regioni diverse e a seconda della funzionalità, è effettivamente una struttura portante in metafase.
Quindi la cromatina interfasica ha una struttura ad anse.

Esistono delle evidenze della struttura ad anse: immagini in microscopia elettronica, esperimenti di sedimentazione, esperimenti di digestione con nucleasi. Tutti queste esperimenti hanno permesso di caratterizzare gli elementi di questa struttura.\\
Le anse sono delle \textbf{unità funzionali}: ci sono diverse prove sperimentali che dimostrano che le anse sono più o meno definibili come delle unità funzionali e cioè delle unità di trascrizione e di replicazione (eventi concertati). 
La struttura ad anse varia a seconda delle esigenze funzionali della regione cromosomica in un certo momento dello sviluppo e in un certo tipo cellulare. Ha un ruolo fondamentale nella regolazione dell’espressione genica e cioè nella trascrizione.

Regola generale, ma non sempre è così: tutto ciò che è attivamente trascritto è anche replicato all’inizio della fase S del ciclo cellulare, tutto ciò che è silenziato è invece replicato alla fine della fase S del ciclo cellulare.
Quello che è altamente condensato è poco trascritto ed ha replicazione tardiva. Quello che è poco condensato è attivamente trascritto ed ha replicazione precoce (durante la fase S iniziale). Questo avviene proprio perché trascrizione e replicazione sono processi metabolici interconnessi.

Lo scheletro proteico che è costituito dalla \textbf{topoisomerasi II}, la quale rappresenta la seconda proteina per abbondanza in massa nel nucleo delle cellule eucariotiche.

L’\emph{asse dell’ansa} è dunque il DNA complessato con gli istoni, mentre la \emph{base dell’ansa} è l’impalcatura che abbiamo visto bene nelle immagini di microscopia elettronica, chiamata \textbf{chromosome scaffold} (matrice del cromosoma) e costituita fondamentalmente dalla topoisomerasi II.

Che la base delle anse sia formata da delle topoisomerasi non è un caso, infatto queste sono proteine fondamentali durante la replicazione e la trascrizione perchè capaci di indurre tagli a singolo e a doppio filamento fondamentali per rilassare le tensioni torsionali che si vengono a creare quando il DNA si decondensa e quando la doppia elica si svolge. Se la doppia elica venisse semplicemente svolta ed aperta si sviluppere una tensione che potrebbe causare delle rotture casuali della molecola di DNA: affichè questo non avvenga ci sono le topoisomerasi che inducono dei tagli mirati a singola elica che vengono poi saldati e servono ad evitare che queste tensioni torsionali portino ad anomalie e ad errori nella replicazione e nella trascrizione.

Il fatto che la topoisomerasi II si trovi alla base delle anse ha un senso se è vero che le anse sono delle unità di trascrizione e di replicazione. La topoisomerasi ha dunque sia una funzione strutturale che enzimatica.
   
 Inoltre l'organizzazione a loops ha un significato molto importante perché ovviamente, se noi creiamo un’ansa anche molto grande, facciamo in modo che regioni fisicamente molto lontane sulla molecola lineare di DNA vengano a trovarsi invece molto vicine.\\
L’organizzazione a loops è dunque in grado anche di tenere fisicamente vicine regioni che devono essere trascritte o replicate nello stesso momento funzionale: queste regioni potranno trovarsi fisicamente vicine per rispondere agli stessi segnali regolatori.

A questo punto, siccome il loop ha un significato regolativo, è facile capire come la dimensione e il grado di condensazione dei loop varino in diversi stadi dello sviluppo e in diversi tessuti.

\begin{wrapfigure}{r}{0.62\textwidth}
    \includegraphics[width=0.60\textwidth]{img/26_anse.jpg}
  \caption{La topoisomerasi II e l'organizzazione ad anse}
\end{wrapfigure}

Il disegno a lato ci fa capire come la struttura a loops possa essere facilmente regolata e cioè come, grazie al legame della base delle anse con la topoisomerasi, siamo in grado di modulare la dimensione e il grado di affastellamento delle anse.\\
Nel disegno la topoisomerasi è rappresentata come una sorta di vagoncino con un occhiello: la parte rossa del vagoncino è quella su cui si trova la regione della topoisomerasi affine al DNA, e cioè la regione di interazione DNA-proteina: la parte rossa è legata alla base delle anse.\\
Il chromosome scaffold è dato dall’allineamento delle diverse molecole di topoisomerasi.

La topoisomerasi quindi è in grado di formare legami DNA-proteina che si trovano alla base dell’ansa, ma anche legami proteina-proteina che saranno i legami tra gli occhielli di molecole di topoisomerasi adiacenti. È evidente come questo permetta facilmente di modificare sia la dimensione che il grado di affastellamento dei loop.

Come faccio a modificare la dimensione dei loop?
Se ho due loop adiacenti, uno di 112 Kb e l’altro di 26 Kb, e tolgo la molecola di topoisomerasi compresa tra i due loop, succede che da due loop di quelle dimensioni ne ottengo uno più grande la cui dimensione sarà esattamente di 112 + 26 Kb.\\
Modulando semplicemente l’interazione DNA-proteina posso modulare la dimensione dei loop, mentre Per quanto riguarda il grado di condensazione è evidente che gli occhielli potranno interagire tra loro in modo lasso o stretto. La forza dell’interazione tra molecole di topoisomerasi adiacenti modulerà il grado di affastellamento dei loop e quindi il loro grado di condensazione.

Abbiamo detto che regione dove il DNA interagisce con la topoisomerasi si trova alla base dei loop, il DNA però può essere o meno legato alla topoisomerasi. Le regioni di DNA capaci di legarsi alle topoisomerasi si chiamano \textbf{SAR (Scaffold Associated Region)}.

Ciò che modula l'organizzazione ad anse è dunque l’interazione più o meno forte tra topoisomerasi (affastellamento), ma anche il legame o meno con la topoisomerasi (dimensione dei loop).

Ma se le regioni SAR sono sempre presenti, perché a volte legano la topoisomerasi  mentre altre volte no? È evidente che le stesse regioni SAR devono essere capaci o meno di legare la topoisomerasi, e questa capacità è determinata dalle modificazioni epigenetiche.\\
Epigenetica vuol dire che la stessa sequenza (o sequenze molto simili) può all’occorrenza trovarsi in una condizione tale da poter interagire con un fattore (in questo caso con una proteina) o meno.

La stessa molecola di topoisomerasi trova delle SAR disponibili o delle SAR incapaci grazie alla presenza di segnali chimici che rendeno le molecole di topoisomerasi in quella regione più affini rispetto a tutte le altre molecole di topoisomerasi. \emph{(la modificazione epigenetica è a livello delle SAR o della topoisomerasi??)}\\
Tutti questi sono segnali epigenetici perché la sequenza del DNA è sempre la stessa. Ad intervenire sono delle piccole molecole in grado di modificare localmente la struttura del DNA (in questo caso della SAR) rendendola aperta o chiusa, oppure di modificare la proteina rendendola più o meno affine alle altre proteine della sua stessa categoria.
Oggi sappiamo che queste piccole molecole sono RNA non tradotti. 

Questi cambiamenti epigentici sono essenziali se pensiamo che ogni loop è un'unità funzionale a sé stante. Nell'organizzazione del DNA possiamo fare una distinzione tra i geni housekeeping (geni che regola il metabolismo di base e che sono attivi in tutte le cellule) e geni tessuto-specifici.\\
I \textbf{geni housekeeping} regolano il metabolismo cellulare e sono trascritti in tutte le cellule ad un livello basale, mentre i geni tessuto-specifici vengono trascritti solo in certi tipi cellulari e sono trascritti a livelli molto più elevati.
A causa delle diverse esigenze di trascrizione l'organizzazione in loop di questi geni sarà diversa:
\begin{itemize}
\item i \textbf{geni housekeeping}, in generale, si trovano in loop piuttosto grandi e abbastanza condensati perché devono essere trascritti in maniera basale in tutte le cellule di tutti gli organismi. Nelle zone di \emph{eterocromatina costitutiva}, completamente inerti, troviamo loop molto grandi e molto molto condensati.
\item i \textbf{geni tessuto-specifici} invece si troveranno in loop piccoli e molto lassi (i.e. loop in cui si trova fondamentalmente un solo gene che viene continuamente trascritto). Quella stessa regione nella cellula di un altro tessuto sarà ipercondensata.
\end{itemize}

Se nei loop piccoli troviamo un solo gene o quasi e di conseguenza la trascrizione risulta molto rapida, i loop grandi conterranno unità trascrizionali che, venendo trascritte in sequenza, necessiteranno di tempi più lunghi.
Bisogna sempre ricordare che, durante la trascrizione e la replicazione, la doppia elica del DNA si deve aprire e i nucleosomi si devono staccare in maniera transiente per poi riattaccarsi subito dopo perché il DNA deve essere accessibile al complesso trascrizionale o al complesso replicativo. Affinchè questo avvenga, la cromatina deve essere in una conformazione adatta: si devono rapidamente staccare gli 8 istoni che formano il nucleosoma ma, appena completato il processo di trascrizione e di replicazione, gli istoni dell’ottamero si devono riassociare al DNA e tutto questo è modulato dalla struttura ad anse.
Le anse sono più o meno compatte e più o meno grandi nei diversi momenti funzionali e in diverse regioni cromosomiche: questo è variabile per le esigenze del differenziamento e dello sviluppo.

Il significato funzionale dei loop è dimostrato anche dal fatto che le SAR normalmente si trovano vicine ai promotori: facendo un trattamento con enzimi di restrizione, considerando anche elementi regolativi a lunga distanza come gli enhancers, è molto facile che sullo stesso frammento di restrizione si trovino sia una SAR che un elemento regolativo a lunga distanza (enhancer).

Cosa sono le modificazioni epigenetiche?\\
Sono cambiamenti della cromatina che regolano l’espressione genica a livello trascrizionale.
Tra i vari livelli di regolazione dell'espressione genica quello più economico è quello che interviene già a livello della trascrizione in quanto evita sprechi metabolici ed energetici.

Le modificazioni epigenetiche non alterano la sequenza del DNA ma la struttura tridimensionale della cromatina, hanno un ruolo chiave nei processi di regolazione dell’espressione genica negli eucarioti e rappresentano il sistema più usato per regolare l’espressione genica.
Le modificazioni epigenetiche non riguardano promotori, enhancers e silencers di per sé, ma piuttosto l’accessibilità di tali sequenze ai fattori con cui interagiscono. Le modificazioni epigenetiche fanno in modo che le sequenze regolatrici siano o meno capaci di interagire con i propri effettori.

Queste modificazioni consistono fondamentalmente in:
\begin{itemize}
\item metilazione del DNA;
\item metilazione ed acetilazione di specifici gruppi degli istoni che formano l’ottamero.
\end{itemize}

Una cosa fondamentale è che in ogni regione del genoma degli eucarioti superiori esiste un rapporto ben preciso tra grado di metilazione e grado di acetilazione e addirittura questo rapporto locale tra numero di acetili e numero di metili rappresenta una sorta di \textbf{codice istonico} che si sovrappone al codice genetico, proprio perché ha la capacità di modulare l’espressione genica locale. In qualche caso il codice istonico è stato decodificato: ci sono delle regioni del genoma in cui si è stabilito esattamente qual è il rapporto chiave tra metili e acetili e quindi il codice istonico per avere un certo livello di espressione genica. 

Le SAR sono state facilmente isolate e caratterizzate trattando il DNA con la stessa sostanza usata per ottenere il back bone proteico (litio 3’, 5 diiodiosalcilato, LIS), ovvero una molecola capace di rimuovere gli istoni. Rimuovendo gli istoni vi saranno delle regioni che risulteranno protette dal trattamento con enzimi di restrizione perché schermate dalla topoisomerasi. A questo putno posso rimuovere la topoisomerasi e caratterizzare i segmenti di DNA che erano stati protetti dalla digestione enzimatica: in questo modo sono state sequenziate le SAR.
Tutti questi esperimenti sono stati fatti fondamentalmente in Drosophila melanogaster.
Questo permette anche misurare la dimensione dei segmenti che si trovano tra due regioni protette dalla digestione enzimatica e quindi valutare la dimensione dei loops e dunque la distanza tra SAR adiacenti. In Drosophila la dimensione dei loops è molto variabile: da 4,5 a 115 Kb.


Tutte le SAR studiate contengono da 8 a 17 box ricorrenti: per \emph{``box''} si intende un blocco di sequenza: questi blocchi sono specifici e ricorrenti, molto simili tra loro in SAR diverse.\\
In tutte le SAR è presente un \emph{sito di legame per la topoisomerasi} costituito dalla sequenza \textbf{GNT(A/T)A(T/C)ATTNATNN(G/A)}. 
Vi sono poi altre due box ricorrenti (tra quelle trovate confrontando le diverse SAR della Drosophila melanogaster) dal significato regolativo: sono presenti o assenti a seconda che ci troviamo in una regione di geni housekeeping (che non devono essere regolati perché son trascritti in tutte le cellule di tutto l’organismo) o di geni tessuto-specifici (regolati). Abbiamo quindi una \textbf{T-box} che è una \textbf{sequenza ricca in T} ed una \textbf{A-box} che è \emph{ricca in A}. La T-box si trova a valle della A-box. La differenza tra geni housekeeping e geni  tessuto-specifici sta proprio in queste box regolative:
\begin{itemize}
\item i geni housekeeping non sono regolati perchè trascritti a livello basale in tutti i tessuti e in tutti gli stadi del differenziamento, hanno delle SAR costitutive sempre legate alla topoisomerasi che mancano della A-box.
\item i geni tessuto-specifici invece presentano SAR che devono essere regolate perchè trascritti ad alto livello soltanto in certi tessuti e presentano la A-box.
\end{itemize}



\chapter{La localizzazione dei geni sui cromosomi}
L’importanza di localizzare i geni sui cromosomi costruendo delle mappe genetiche non è solo di tipo topografico. 
Le prime mappe sono state costruite ordinando sui cromosomi dei geni noti. 
Uno degli esperimenti storici di mappaggio dei geni è stato fatto in Drosophila, ed il primo cromosoma ed essere stato mappato è stato l’X. Questi esperimenti sono stati fatti con un approccio di tipo genetico calcolando la \emph{distanza dei geni sulla base delle frequenze di ricombinazione}. Il metodo che venne utilizzato fu quello dell’incrocio a tre punti, il quale permette di stabilire in base alla frequenza di ricombinazione la distanza dei geni. La frequenza di ricombinazione di due geni è proporzionale alla distanza e dunque alla probabilità che questi vengano separati da un evento di ricombinazione: è un \emph{approccio di tipo statistico}.
Poichè il metodo statistico non ci da un’informazione sulla distanza lineare (i.e. numero di paia di basi) tra due geni, questa può essere trovata anche tramite un approccio di tipo fisico.

Mappare i cromosomi significa soprattutto identificare nuovi geni, e identificare la distanza tra due o più geni ci permette sia di capire come questi interagiscono tra loro ma anche di identificare segnali regolativi che ne influenzano l’espressione.

Al significato topografico si aggiungono degli elementi che permettono di fare delle deduzioni sul funzionamento di questi geni in base al loro rapporto di linkage: l’identificazione di blocchi di geni altamente conservati e nello stesso linkage lascia presupporre che questi ultimi abbiano attività estremamente correlate tra loro e che abbiano anche avuto un importante significato evolutivo.

Ad oggi grazie agli approcci post-genomici è possibile allineare su un genoma degli enormi segmenti di DNA, e ciò su cui ci si sta concentrando è cercare di stabilire i ruolo di singole sequenze, come queste interagiscano le une con le altre e come rispondano agli stessi segnali regolativi. Ad oggi è possibile arrivare fino alla determinazione della sequenza con Sanger oppure con gli approcci di NGS (rapido sequenziamento di genomi complessi).

A cosa serve costruire una mappa?
\begin{enumerate}
\item conoscere la topografia del genoma consente di \textbf{prevedere e controllare l’eredità dei caratteri}
\item si possono costruire dei modelli di regolazione dell’espressione genica basati sulla presenza di sequenze regolatrici, ma si possono anche \textbf{costruire dei modelli di regolazione epigenetica} proprio perché il fatto che dei geni si trovino in particolari domini genomici fa in modo che questi abbiano un tipo di regolazione simile e quindi rispondano in maniera simile agli stessi segnali regolatori
\item \textbf{studiare le interazioni tra i geni}
\item \textbf{studiare l’organizzazione del genoma}: nell’era post-genomica, dopo il sequenziamento grezzo del genoma (terminato nel 2001), si è aperto un nuovo mondo grazie alla scoperta di tantissime sequenze che non sono codificanti ma che hanno un ruolo chiave nella regolazione dell’espressione genica
\item \textbf{studiare l’evoluzione dei genomi}: fare un’analisi comparata della topografia di genomi più o meno correlati evolutivamente permette di capire quali siano i meccanismi molecolari che governano l’evoluzione dei genomi.
\end{enumerate}

Questi sono approcci fondamentali per completare le informazioni statistiche che vengono dal mappaggio dei genomi con le tecniche di NGS, che hanno l’enorme limite di essere approcci statistici che vanno verificati. A tal proposito di si parla si \textbf{dry laboratory approches} e \textbf{wet laboratory approches}. I primi, quelli in asciutto, sono quelli fatti con approcci bioinformatici analizzando banche dati, si basano su studi probabilistici che hanno bisogno di verifiche in bagnato, cioè sperimentali, in laboratorio.



\section{Metodi di localizzazione genica nell'uomo}

Vediamo una panoramica dei metodi classici:
\begin{enumerate} 
\item \textbf{Mappaggio genetico (statistico)}
\begin{itemize}
\item \textbf{Analisi di linkage}. Queste analisi non sono soltanto utilizzate per l’identificazione dei geni malattia ma anche per capire se alcuni geni malattia sono più o meno correlati ad altri geni che ne influenzano la funzione (per questo è molto importante lo studio degli alberi genealogici). Lo studio di linkage non è altro che il trasferimento ad \emph{organismi nei quali \underline{non è possibile} fare incroci programmati} delle analisi di linkage classiche, che studiano i rapporti fra i geni e la loro distanza sulla base delle frequenze di ricombinazione. È evidente che un approccio del genere, nella Drosophila o nel mais, prevede la conoscenza di marcatori morfologici e la programmazione di incroci. In questi casi il numero di caratteri morfologici che possono essere studiati è enorme, è possibile programmare degli incroci e selezionare delle linee pure attraverso numerose fasi di incrocio (ovvero incrocio fra fratelli) per poi andare a studiare una progenie molta numerosa (numeri statisticamente rilevanti) con tempi di generazione che sono di alcune settimane. È evidente come questo metodo non possa essere trasferito ad organismi superiori (e.g. uomo e/o organismi di interesse zootecnico come i bovini) non solo perché è più complicato programmare gli incroci, ma anche per i tempi di generazione e la numerosità della progenie. Inoltre nell’uomo queste analisi non possono essere condotte non solo per i tempi di generazione (30 anni) ma anche perché il numero di caratteri morfologici è esiguo: negli studi di linkage è vero che nell’uomo si studierebbe la segregazione di geni malattia, ma lo studio di linkage prevede di studiare la co-segregazione del gene che ci interessa con un altro marcatore fenotipico facilmente identificabile e siccome nell’uomo non si può parlare di colore del pelo o cose del genere, questo è molto difficile.
 
Quindi, non potendo programmare gli incroci, le analisi di linkage nell’uomo consistono nello studiare degli incroci già avvenuti: \emph{analisi dei pedigree}. Questo tipo di analisi si fa su grandi o piccole famiglie purché si abbiano tante famiglie in cui segrega la stessa malattia e per poi analizzare la co-segregazione della malattia con dei marcatori di riferimento. Come marcatori si possono utilizzare diversi marcatori proteici, ad esempio oggi si usano marcatori del DNA che consistono fondamentalmente nei polimorfismi del DNA come ad esempio i polimorfismi RFLP, e cioè polimorfismi che riguardano la lunghezza di frammenti di restrizione: se si digerisce il DNA con un enzima di restrizione si ottiene una sonda che identifica quel segmento. Esistono dei polimorfismi che non hanno effetto fenotipico ma modificano la dimensione di questi frammenti, caratteristica di un individuo. Se un certo polimorfismo co-segrega sempre con il gene malattia avrò stabilito qual è l’aplotipo, ovvero il cromosoma che ha il gene malattia e questo mi permette di seguire la segregazione del cromosoma nella famiglia e fare delle deduzioni sulla probabilità che un individuo sia portatore del gene malattia (per fare poi della consulenza genetica). In realtà oggi non si usano quasi più gli RLFP ma si usano mutazioni anche di singoli nucleotidi, oppure polimorfismi che riguardano minisatelliti e microsatelliti. All’inizio quindi si usavano i polimorfismi proteici, oggi si usano i polimorfismi del DNA, microsatelliti, minisatelliti e SNPs. La citogenetica non si limita a determinare la distanza tra i geni e la probabilità che vengano trasmessi, ma è un approccio fisico.
\end{itemize}

\item \textbf{Mappaggio citogenetico molecolare}. Con questo approccio si riesce esattamente a stabilire dove si trova il gene sul cromosoma e a che distanza si trova da un particolare gene marcatore che ci interessa. Gli approcci di cui parleremo sono quelli di: 
\begin{itemize}
\item \textbf{mappaggio citologico} (sui cromosomi)
\item \textbf{studio della dose genica}, che ormai non si usa più
\item \textbf{ibridazione di cellule somatiche} - non solo è un approccio per la localizzazione genica molto potente ma è anche utile per spiegare come ibridi somatici siano quelli che permettono di produrre Ab monoclonali che sono il presente e il futuro della ricerca soprattutto nel campo della diagnostica precoce della terapia del cancro.
\item \textbf{ibridazione in situ}. Questa tecnica che negli ultimi 15-20 è arrivata a una potenzialità descrittiva che permette di sovrapporre questo approccio a quello del mappaggio fisico molecolare che consiste fondamentalmente nella costruzione di mappe di restrizione fino ad arrivare all’allineamento di Bach per il sequenziamento grezzo (detto draft genome sequencies). Le prime sequenze complete dei genomi vengono fatte attraverso l’allineamento di grossi contigui fino ad arrivare al sequenziamento vero e proprio. 
\end{itemize}
  
\item \textbf{mappaggio fisico} (molecolare, sulla sequenza):
\begin{itemize}
\item \textbf{mappe di restrizione} classiche
\item \textbf{mappe di restrizione “long range"}
\item \textbf{mappe di cloni di DNA contigui}
\item \textbf{sequenziamento del DNA}
\end{itemize}
\end{enumerate}

Le tecniche di ibridazione in situ, di ibridazione di cellule somatiche e di mappaggio fisico sono altamente complementari. Le prime due arrivano a una capacità di risoluzione che si avvicina alla singola base e dà informazioni anche di tipo architettonico (cosa c’è sopra, sotto, destra ecc..)

\subsection{Le analisi di linkage}
Questa tecnica consiste nello studiare nelle famiglie la co-segregazione di un gene malattia e di un marcatore. È una tecnica utile per identificare l’aplotipo, cioè la sequenza sul cromosoma legata alla trasmissione della malattia. Questo permetterà di studiare i polimorfismi del DNA e in base alla presenza o assenza di un polimorfismo si può capire se l’individuo è portatore del gene malattia o meno. Ciò è importante per scopi diagnostici, per la consulenza genetica e per lo studio nelle popolazioni dell’incidenza di una particolare mutazione. Nello studio di malattie ereditarie ma anche del cancro ci permetterà di sapere qual è la mutazione più frequente in quella particolare popolazione, e quindi cercare prima quella.

\subsection{Ibridazione di cellule somatiche} 
Nei mammiferi le cellule somatiche sono così altamente differenziate che difficilmente si adattano ad essere coltivate in vitro (non si replicano o lo fanno occasionalmente in risposta a un qualche stress o danno). Questi approcci complessivamente prendono il nome di \emph{``genetica di cellule somatiche''}.

Mentre con organismi quali lieviti e batteri è facile fare incroci e riconoscere rapporti di dominanza grazie agli svariati marcatori metabolici e alle capacità proliferative di queste cellule, nei mammiferi questo approccio è molto più difficile e presenta svariati limiti.\\
Fu il Prof. Pontecorvo a notare per primo che anche le cellule di mammifero possono andare incontro ad eventi di fusione (evento comune in procarioti e eucarioti unicellulari ma molto raro nei mammiferi). A seguito di questa osservazione vennero messi in coltura fibroblasti sottocutanei, i quali sono capaci di adattarsi e crescere in vitro come cellule isolate. Una volta che queste cellule sono adese ad un supporto (sono cellule che provengono da un tessuto solido, hanno dunque bisogno di una sorta di membrana basale a cui aderire per crescere, per questo vengono coltivate in piastra Petri) possono essere amplificate e congelate come colture cellulari.\\
A seguito di questa osservazione si pensò che fosse possibile considerare le colture di cellule somatiche come popolazioni di singole cellule, e che ogni cellula potesse essere considerata come un singolo organismo. Poichè ogni cellula possiede tutta l’informazione genetica dell’organismo da cui deriva si pensò di promuovere la capacità di fusione spontanea di queste ultime per poi analizzare l’ibrido, ottenendo così un risultato simile a quello di un incrocio.\\
È ovvio che questa tecnica presenta dei limiti rispetto ai batteri: non posso studiare molti caratteri morfologici ma ci sarà un limite a quelli che posso studiare perché si esprimono in cellula, sono caratteristiche morfologiche ma soprattutto metaboliche (espressione di marcatori proteici e forma della cellula per esempio). La genetica di cellule somatiche nasce dunque da quest’idea.

Viene definita \textbf{eterocarionte} una cellula che possiede: una sola membrana plasmatica, un solo citoplasma, due nuclei. Questa è una cellula eterogenea per quanto riguarda il carion, ovvero i nuclei.
A volte può succedere che anche i nuclei si fondano portando alla formazione di una nuova cellula nella quale il nucleo contiene entrambi i genomi completi delle cellule parentali. Questa sarà una cellula tetraploide detta \textbf{sincarionte}.

La fusione cellulare avviene spontaneamente in alcuni tessuti (e.g. nel fegato dove ha un significato importante perché si ottengono cellule poliploidi ad alta secrezione, nelle ghiandole e nel muscolo scheletrico, dove troviamo dei sincizi).

Poichè quello della fusione è un fenomeno raro nei mammiferi, sono state studiate delle tecniche che potessero aumentarne la frequenza. A questo scopo è stato studiato il \textbf{virus Sendai} (Sendai è una città del Giappone, il virus fu scoperto nel 1965), il quale è conosciuto anche come Virus Emoagglutinante del Giappone (HVJ). Questo virus ha la capacità di aumentare la fusione delle vescicole lipidiche e delle membrane cellulari collegando fra loro due membrane e creando un vero e proprio ponte citoplasmatico tra le due cellule (ovviamente queste si devono trovare fisicamente vicine). Il virus viene isolato da uova di pollo gallate (fecondate, le uova gallate poi marciscono) e poi inattivato tramite esposizione a luce UV. Ad oggi questo virus non è più utilizzato ma viene usato un polimero, con catena più o meno estesa, che è il PEG.

Il \textbf{PEG (Glicole Poli Etilenico)} è un polimero preparato per \emph{polimerizzazione dell'ossido di etilene}. Ne esistono diversi tipi classificati in base alla lunghezza media delle molecole (i.e. diverso PM. La scelta del peso molecolare dipenderà dalle cellule che abbiamo).\\
I diversi polimeri hanno differenti proprietà fisiche (e.g. la viscosità), inoltre sono un po' citotossici (ma non alle dosi usate per la fusione cellulare). Il PEG è affine alla membrana cellulare e idrosolubile, crea ponti citoplasmatici fra le cellule in contatto fisico e in questo modo promuove la fusione delle membrane cellulari.

Per creare gli ibridi si prendono le due cellula parentali e le si miscelano in un rapporto 1:1, dopodichè alla concentrazione opportuna di cellule si aggiunge la concentrazione opportuna di PEG. Inizialmente si formeranno degli eterocarionti e poi dei sincarionti.

Gli ibridi somatici possono essere:
\begin{itemize}
\item \textbf{intraspecifici}. In questo caso vengono prodotti fondendo cellule in coltura provenienti da organismi della stessa specie che saranno diversi per certi marcatori metabolici o comunque marcatori che si esprimono in coltura cellulare.\\
Questi incroci sono \textbf{stabili in coltura} e sono cellule \textbf{tetraploidi}.

\item \textbf{interspecifici}. Vengono prodotti fondendo cellule in coltura provenienti da organismi di specie diverse.\\
Questi ibridi sono utilizzati per la localizzazione dei geni sui cromosomi.\\
In questo caso all’inizio si forma un sincarionte, ma poiché sono cromosomi che provengono da specie diverse questi risultano \textbf{instabili in coltura}. In questo caso c’è una tendenza a ripristinare la condizione originaria di una delle due cellule parentali (viene ripristinata la condizione ``parentale dominante'', più forte). Questo dipende dalla forza dell’impianto che forma il citoscheletro e di conseguenza dalla forza del fuso mitotico. L’impianto cellulare che risulterà essere dominante sarà quello della cellula parentale i cui centromeri hanno, in una competizione interna fra i centromeri di una specie e centromeri dell’altra, una maggiore affinità per le proteine del cinetocore e risultano dunque più abili nell’attaccarsi al fuso mitotico. Questa competizione fra centromeri fa sì che l’impianto cellulare vincente mantenga tutti i cromosomi, mentre vi sarà una perdita progressiva e casuale dei cromosomi dell’altra specie. È questa caratteristica che li rende importanti negli studi di localizzazione genica.
\end{itemize}

Con l’analisi di ibridi intraspecifici:
\begin{itemize}
\item posso vedere se cellule che esprimono marcatori metabolici diversi, una volta fuse, esprimono il marcatore dell’uno o dell’altra cellula parentale e vedere quindi se c’è un repressore o un induttore, stabilendo i \textbf{rapporti di dominanza tra diverse mutazioni}. Questo significa anche fare dei test di complementazione, cioè vedere se la mutazione presente in una delle due cellule parentali è allo stesso locus (in questo caso non c’è complementazione), mentre se la mutazione è presente in loci diversi allora nell’ibrido ci sarà complementazione e quindi avrà un fenotipo normale
\item studio dei \textbf{rapporti di dominanza tra geni} e quindi anche capire se ci sono meccanismi di repressione o induzione dell’espressione genica 
\item \textbf{produzione di anticorpi monoclonali}, utili soprattutto per la terapia personalizzata
\end{itemize}

Esistono anticorpi:
\begin{itemize}
\item \textbf{policlonali}: ogni linea clonale di cellule B secerne nel siero anticorpi diretti contro un singolo epitopo. Nel siero di un animale immunizzato contro un agente patogeno si trova una miscela di anticorpi poiché esistono più linee B, ognuna delle quali produrrà anticorpi contro un singolo epitopo di un agente immunizzante, dunque quello che purifico dal siero di un animale immunizzato sarà un anticorpo policlonale, ovvero una miscela di Ab prodotti da più plasmacellule (linee B) indotte a produrre Ab dal contatto con l’agente patogeno.
      
\item \textbf{monoclonali}: sono prodotti da una singola linea clonale di linfociti B e sono diretti contro un singolo epitopo. Dopo aver immunizzato l’animale devo riuscire a isolare singole linee B ciascuna della quali produce nel terreno di coltura un Ab diretto contro un singolo epitopo. Questo è fondamentale quando si ha a che fare con agenti antigenici molto simili tra loro ma diversi per singoli epitopi. Ciò succede anche tra virus simili tra loro, o nelle cellule tumorali che hanno questa caratteristica e ciò le rende particolarmente difficili da caratterizzare, perché ogni tumore di ogni individuo è un caso a sé, quindi cellule di uno stesso tipo tumorale condivideranno alcune caratteristiche ma in realtà ogni individuo avrà una sua particolare unica linea di cellule tumorali per un particolare tumore. Questo è un problema non tanto per la diagnosi ma più che altro per la terapia.
\end{itemize}

\textbf{Epitopo}: singolo elemento che genera la risposta immunitaria, ovvero la produzione di uno specifico Ab. Ogni molecola complessa ha tanti epitopi e ogni linea B produce un Ab diretto contro uno di questi epitopi. L’epitopo è la parte di antigene che lega l'anticorpo specifico. La singola molecola di antigene può contenere diversi epitopi riconosciuti da anticorpi differenti. L’immunizzazione permette di purificare una miscela di Ab che saranno efficienti contro l’antigene perché reagiscono contro tutti gli epitopi dell’antigene che ho usato per immunizzare l’animale. 


Gli anticorpi hanno una variabilità di titolo e di reattività tra uno stock e l’altro, questo perché immunizzo un animale ma poi devo vedere nel siero qual è effettivamente il titolo anticorpale e l’affinità dell’Ab contro l’antigene in studio. A seconda dell’animale e della procedura può esservi una variabilità enorme del titolo anticorpale nel siero. L’utilizzo del siero non è utile per saggi diagnostici proprio per i motivi che abbiamo appena detto.\\
Gli anticorpi monoclonali sono prodotti da singoli cloni di linfociti B, sono altamente specifici e possono essere prodotti illimitatamente e in modo assolutamente riproducibile. Facendo degli ibridi intraspecifici invece creo delle colture di cellule immortalizzate capaci di produrre solo quel tipo di Ab. Queste sono cellule trasformate e vengono dette ``immortali'' perché nelle condizioni opportune non hanno un tempo di replicazione determinato come invece avviene in tutte le colture di cellule differenziate. Ad esempio i fibroblasti umani fanno al massimo 50 generazioni in vivo (in coltura meno) mentre le cellule immortalizzate possono essere riprodotte e amplificate all’infinito, non perdono le loro potenzialità replicative e sono ideali per saggi diagnostici e altre applicazioni.

\clearpage
\subsubsection{Terapia con anticorpi monoclonali}
Questa terapia è promettente in quanto:
\begin{itemize}
\item l'uso di anticorpi monoclonali che si legano specificamente alle cellule bersaglio, permette di stimolare il sistema immunitario del paziente ad aggredire ad hoc solo le cellule riconosciute (questa è la chiave del premio Nobel di quest’anno: rendere una cellula che normalmente non sarebbe riconosciuta, non sarebbe attaccata dal sistema immunitario, riconoscibile)
\item è possibile creare un Ab specifico per quasi tutti gli antigeni di superficie extracellulare delle cellule bersaglio
\item esistono anticorpi attivi contro malattie gravi quali l'artrite reumatoide, la sclerosi multipla e diverse forme di tumore
\item in particolare, nei tumori, permette di creare delle terapie personalizzate.
\end{itemize}

Esistono due tipi di anticorpi monoclonali:
\begin{itemize}
\item \textbf{anticorpi monoclonali nudi} senza alcun farmaco o materiale radioattivo ad essi chimicamente legato per stimolare la risposta immunitaria
\item \textbf{anticorpi monoclonali coniugati} quando sono uniti a un farmaco chemioterapico, a un isotopo radioattivo o una tossina citotossica. Se vogliamo fare della radioterapia mirata coniugo l’Ab con il radioisotopo e in questo modo, invece di andare a colpire in maniera indiscriminata tutte le cellule di una certa porzione di tessuto (tumorali e non), vado a colpire solo la cellula che mi interessa. Questo permette di evitare gli effetti collaterali. Posso anche usare una tossina citotossica che ammazza la cellula tumorale.
\end{itemize}
      
Nel momento in cui individuo dei marcatori tumorali posso identificare singole cellule in maniera molto efficiente marcandole con un particolare Ab monoclonale. Questo mi permette eventualmente di isolarle usando tecniche di sorting specifico (citofluorimetri separatori). Tramite il citofluorimetro a flusso separatore è possibile identificare anche singole cellule in un tessuto sano, e quindi vedere precocemente se le cellule tumorali hanno infiltrato un certo tessuto e prevenire la diffusione di metastasi.

Come si costruiscono gli ibridomi per la produzione di anticorpi monoclonali? 
Si parte dalla costruzione di cellule ibride intraspecifiche, ovvero date dalla fusione di cellule provenienti dalla stessa specie.\\
La logica è molto semplice e si basa sul concetto di \textbf{complementazione genetica}: si prendono due linee cellulari che hanno caratteristiche diverse, le si mette insieme e si ottiene una nuova linea cellulare, l’ibridoma, che somma in sé stessa le caratteristiche di entrambe le linee cellulari parentali. Ovviamente si devono eliminare tutte le cellule parentali per avere alla fine esclusivamente le cellule che conservano per complementazione le caratteristiche di entrambe le linee cellulari.\\
Nello specifico in questo caso si prendono dei topi e li si trattano con l’agente patogeno immunizzandoli, dopodiché si preleva la milza e si ottengono cellule linfocitarie di topo capaci di produrre e rilasciare nel terreno di coltura degli anticorpi. Queste cellule però daranno un mix di anticorpi policlonali in quanto la milza è costituita da diverse linee B.\\
A questo punto i linfociti della milza vengono fusi con delle cellule tumorali (linee stabilizzate e immortalizzate di topo) derivanti da \emph{linfomi} (tumori derivati da cellule del sistema ematopoietico). Fondendo questi due tipi di cellule ottengo l’ibridoma, ovvero un unico tipo cellulare che possiede sia la capacità derivante dai linfociti di produrre anticorpi, che la capacità di replicarsi in maniera indefinita derivante dalle cellule tumorali. Si avrà comunque una coltura mista che produrrà una miscela di diversi Ab.

Se prendessimo delle cellule primarie di linfociti o fibroblasti da un espianto di tessuto differenziato otterremmo delle cellule poco efficienti nella replicazione in piastra in quanto queste cellule sono abituate a stare in un tessuto, hanno bisogno di essere in tante per avere una cooperazione metabolica. Per questa ragione si dice che sono \textbf{cellule non autonome}.
Le \textbf{cellule immortali} invece non solo crescono in maniera indefinita ma hanno anche autonomia, sono bene adattate alle condizioni di coltura e possono crescere anche come singola cellula, ovvero possono clonare.
Se predo una piastra a pozzetti o Petri (se prendo una piastra Petri le semino molto rade, se prendo una multiwell da 90 pozzetti vado a diluire in modo tale da seminare 0-1 cell per pozzetto - non devo avere mai più di una cellula per pozzetto) e semino le cellule immortalizzate queste crescono e formano dei cloni.

Un volta costruito semino le cellule dell’ibridoma, che hanno la caratteristica i produrre Ab e di clonare, o in piastre o in multiwell e poi amplifico i singoli cloni: ogni clone avrà la caratteristica di produrre l’anticorpo di una sola delle linee B presenti nella milza dell’animale che è stato utilizzato per la fusione. Alla fine avrò dunque tanti cloni, ciascuno dei quali produce un anticorpo monoclonale diretto contro un diverso epitopo.

NB. Ovviamente prima di clonarle si devono selezionare solo gli ibridi: eliminare le cellule parentali di linfociti non è difficile, in quanto queste non sono immortali e non crescono a lungo termine, mentre per eliminare le cellule parentali di linfomi si usa un terreno selettivo (terreno HAT).

Ciascuna coltura produrrà dunque un anticorpo monoclonale diverso che dovrà essere caratterizzato per poi decidere se utile ai nostri scopi o meno.

\begin{wrapfigure}{r}{0.62\textwidth}
    \includegraphics[width=0.60\textwidth]{img/27_ibridi_anticorpi.png}
  \caption{Immunizzazione per produzione di Ab}
\end{wrapfigure}

L’antigene con cui si immunizza il topo potrà essere una cellula intera, una membrana, un microorganismo. L'immunizzazione è facilitata tramite l'utilizzo di alcuni adiuvanti. Si fa una titolazione dell’anticorpo presente nel siero e poi si preleva la milza dopo due settimane.

Le cellule di mieloma (cellule tumorali) utilizzate per la fusione non devono solo essere immortali ma anche presentare la mutazione \textbf{HGPRT-} (mancano dell’enzima HGPRT). Questo enzima è normalmente responsabile della capacità della cellula di \emph{prelevare l’8-azaguanina}, un analogo della guanina e un potente mutageno citotossico, dal terreno di crescita. Questa caratteristica serve per selezionare le cellule ibride.

Il terreno di selezione utilizzato è il \textbf{terreno HAT} sul quale sopravvivono solo le cellule HGPRT+.\\
Perché le cellule HGPRT- muoiono? Perché il terreno HAT contiene un \textbf{inibitore della sintesi endogena dei nucleotidi}. 

HAT sta per:
\begin{itemize}
\item \textbf{hipoxantina}, un precursore delle purine
\item \textbf{aminopterina} (o \textbf{ametopterina}), un inibitore dell'enzima DHFR (deidrofolato reduttasi). Questo è un enzima essenziale per la via di sintesi endogena delle basi e dunque del DNA.\\
Questa è la stessa molecola utilizzata per il \textbf{metotrexato} o metotrexate, uno dei più diffusi agenti antiblastici (utilizzata come agente antitumorale proprio perchè blocca la crescita delle cellule inibendone la sintesi del DNA, è una molecola molto usata nella chemioterapia)
\item \textbf{timidina}, un precursore delle pirimidine; 
\end{itemize}

A causa della presenza dell'aminopterina le cellule seminate su terreno HAT devono essere capaci di prelevare i precursori nucleotidici dal terreno. Le cellule di mieloma tuttavia, a causa della mutazione HGPRT-, non sono capaci di prelevare dal terreno le basi puriniche. Questa selezione può essere fatta anche mutando l'enzima TK (responsabile del prelievo delle basi pirimidiniche dal terreno).\\
Tramite questo terreno posso dunque eliminare le cellule di mieloma che non si sono fuse, mentre quelle che si saranno fuse formando l’ibridoma avranno per complementazione il gene HGPRT della cellula donatrice della milza, e saranno quindi capaci di crescere.

Perciò, inibendo l'enzima DHFR tramite l'aminopterina, non avviene la sintesi de novo delle basi. L’enzima DHFR infatti, attraverso una serie di reazioni, porta alla formazione dell’\textbf{acido N5, N10 metilen-tetraidrofolico}, che è il maggior donatore di metili per la formazione dei precursori per la sintesi del DNA. A causa del blocco della sintesi endogena del DNA viene utilizzata la via di salvataggio per la quale sono indispensabili due enzimi: la \textbf{timidina chinasi (TK)} per l’uptake delle pirimidine e l’\textbf{enzima ipoxantina-guanina-fosfo-ribosil-transferasi (HGPRT)} per l’uptake delle purine.

Ovviamente non si formeranno solo ibridi milza-mieloma ma anche ibridi omocellulari milza-milza, i quali però moriranno in quanto cellule primarie non immortali.

I cloni possono essere espansi anche nel topo, dove produrranno tumori ascitici portando a un alto titolo anticorpale (oggi non si fa più in vivo, vengono usati dei fermentatori).

\begin{figure}[h!]
\centering
\includegraphics[scale=0.65]{img/28_anticorpi_monoclonali.png}
\caption{Preparazione anticorpi monoclonali in ibridi intraspecifici}
\label{}
\end{figure}

\subsection{Gli anticorpi umanizzati}
Per la terapia dei tumori si utilizza la cellula tumorale per immunizzare il topo cosicchè vengano produtti gli Ab con l'antigene tumorale specifico del paziente. 
In questo modo può essere attivata la risposta immunitaria contro la cellula tumorale richiamando le cellule NK, o associando all’anticorpo delle molecole terapeutiche.

È evidente che se lo scopo è terapeutico sarà necessario produrre anticorpi monoclonali umani e non più murini, i quali ovviamente non possono essere prodotti immunzzando un essere umano. Per questa ragione si producono gli \emph{``anticorpi umanizzati''}.\\
Se si somministra a un essere umano un Ab prodotto in una cellula animale diversa questo viene trattato come non-self e contro di esso si scatena una risposta immunitaria.
Nel caso della terapia tumorale però si vuole cercare di far passare anticorpi prodotti in una certa cellula come cellule self.

\begin{wrapfigure}{r}{0.70\textwidth}
    \includegraphics[width=0.7\textwidth]{img/30_ig_umanizzate.png}
  \caption{}
\end{wrapfigure}

Le difficoltà tecniche per produrre in vitro anticorpi umani da utilizzare per la terapia sono tante: non ci sono linee di mieloma adeguate, le cellule B umane possono essere immortalizzate ma perdono la capacità di produrre anticorpi (ovviamente non posso fare un trattamento in vivo ma devo trattare colture cellulari e la coltura cellulare non produce anticorpi a lungo termine) e inoltre è difficile ottenere cellule B umane attivate dall’antigene.

Per questa ragione non si agisce manipolando direttamente gli anticorpi ma producendo degli \textbf{anticorpi ibridi umanizzati}.
Per farli si modifica geneticamente il DNA di topo in modo che le cellule murine sintetizzino anticorpi con le \textbf{porzioni variabili murine} e le \textbf{porzioni costanti umane}.
La \emph{porzione variabile} è quella che riconosce l’anticorpo specifico, mentre la \emph{porzione costante} è quella che verrebbe riconosciuta come non self.

\begin{wrapfigure}{r}{0.50\textwidth}
    \includegraphics[width=0.50\textwidth]{img/31_topi_transgenici.png}
  \caption{}
\end{wrapfigure}

Anticorpi monoclonali umani si possono dunque ottenere producendo topi transgenici nei quali i geni che codificano le catene leggere e pesanti delle Ig di topo sono stati inattivati e sono stati introdotti al loro posto dei geni che codificano le catene leggere e pesanti umane.
In questo modo si ottiene un gene che codifica un anticorpo specifico, ma con un impianto anticorpale da Ig umana.

Naturalmente in questo processo la cosa più complicata è fare la modificazione genetica della cellula ricevente che verrà utilizzata per la produzione degli Ab umanizzati.
La differenza in questo approccio rispetto a quello visto prima è che nel primo lavoro veniva modificata la cellula dell’ibridoma lavorando in coltura cellulare, mentre in questo caso vengono costruiti dei topi transgenici capaci di produrre già Ig umanizzate.




































\end{document}